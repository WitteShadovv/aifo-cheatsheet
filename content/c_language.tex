\section{C}
\subsection{Pointer}
Ein Pointer ist eine Variable, dessen Wert (im Normalfall) die Adresse einer anderen Variablen  ist. Adresse, welche auf ein Objekt zeigt. 
Definition: void* oder int* oder auch int**
\begin{minted}[numberblanklines=true,showspaces=false,breaklines=true]{c}
#include <stdio.h>

int main () {

   int  var = 20;   /* actual variable declaration */
   int  *ip;        /* pointer variable declaration */

   ip = &var;  /* store address of var in pointer variable*/

   printf("Address of var variable: %x\n", &var  );

   /* address stored in pointer variable */
   printf("Address stored in ip variable: %x\n", ip );

   /* access the value using the pointer */
   printf("Value of *ip variable: %d\n", *ip );

   return 0;
}
\end{minted}
\subsection{Conditionals}
Ausdrücke, welche interpretiert werden, dass 0 falsch ist und jeder andere Wert wahr.
\subsection{Referenz- und Dereferenzoperator}
Der Operator «\&» erzeugt die Adresse eines Ausdrucks. \\
Pointer (*) werden jeweils Adressen (\&) zugewiesen, sprich der eigentliche Wert von T* ist \&a.\\
«*» und «\&» heben sich gegenseitig auf» --> *\&a = a = \&*a \\
Achtung bei \&*a muss a Pointer sein, weil die Adresse verlangt wird.
% \subsection{Arrays}
% \colorbox{red!30}{TBD}